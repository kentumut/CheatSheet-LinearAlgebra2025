\subsection*{Gaussian Elimination.}
Given $Ax=b$, form the augmented matrix $[A\,|\,b]$ and apply elementary
row operations to reach row echelon form (REF):
pivot $\rightarrow$ swap $\rightarrow$ eliminate below $\rightarrow$ repeat.
If a row $(0\,\cdots\,0\,|\,c)$ with $c\neq0$ appears, the system is inconsistent;
otherwise solve by back-substitution.
\[
    \left[\begin{array}{cc|c}
            1 & 1 & 3 \\
            2 & 1 & 4
        \end{array}\right]
    \;\xrightarrow{R_2-2R_1}\;
    \left[\begin{array}{cc|c}
            1 & 1  & 3  \\
            0 & -1 & -2
        \end{array}\right]
    \Rightarrow (x,y)=(1,2).
\]
\subsection*{Gauss-Jordan Elimination.}
Starting from $[A\,|\,b]$, apply Gaussian elimination, then normalize each
pivot to $1$ and eliminate all other entries in the pivot columns.
The resulting reduced row echelon form (RREF) gives the solution directly.
Solve:
\[
    \begin{cases}
        x + y = 3, \\
        2x + y = 4.
    \end{cases}
    \qquad\Longleftrightarrow\qquad
    \left[
        \begin{array}{cc|c}
            1 & 1 & 3 \\
            2 & 1 & 4
        \end{array}
        \right]
\]
Row-reduce to RREF:
\[
    \begin{aligned}
        \left[
            \begin{array}{cc|c}
                1 & 1 & 3 \\
                2 & 1 & 4
            \end{array}
            \right]
         & \xrightarrow{\,R_2 \leftarrow R_2 - 2R_1\,}
        \left[
            \begin{array}{cc|c}
                1 & 1  & 3  \\
                0 & -1 & -2
            \end{array}
        \right]                                        \\[6pt]
         & \xrightarrow{\,R_2 \leftarrow -R_2\,}
        \left[
            \begin{array}{cc|c}
                1 & 1 & 3 \\
                0 & 1 & 2
            \end{array}
            \right]
        \xrightarrow{\,R_1 \leftarrow R_1 - R_2\,}
        \left[
            \begin{array}{cc|c}
                1 & 0 & 1 \\
                0 & 1 & 2
            \end{array}
            \right].
    \end{aligned}
\]

\[
    x = 1, \qquad y = 2.
\]

\subsection*{Inverse via Gauss--Jordan.}
To compute $A^{-1}$, form the augmented matrix $[A\,|\,I]$ and apply
Gauss--Jordan elimination. If
\[
    [A\,|\,I] \;\longrightarrow\; [I\,|\,B],
\]
then $B=A^{-1}$. If $I$ cannot be obtained on the left, $A$ is not invertible.
\[
    \begin{aligned}
        [A\,|\,I]1 =
        \left[
            \begin{array}{cc|cc}
                1 & 2 & 1 & 0 \\
                3 & 4 & 0 & 1
            \end{array}
            \right]
        \xrightarrow{\,R_2 \leftarrow R_2 - 3R_1\,}
        \left[
            \begin{array}{cc|cc}
                1 & 2  & 1  & 0 \\
                0 & -2 & -3 & 1
            \end{array}
        \right] \\[6pt]
        \xrightarrow{\,R_2 \leftarrow -\frac12 R_2\,}
        \left[
            \begin{array}{cc|cc}
                1 & 2 & 1        & 0         \\
                0 & 1 & \tfrac32 & -\tfrac12
            \end{array}
            \right]
        \xrightarrow{\,R_1 \leftarrow R_1 - 2R_2\,}
        \left[
            \begin{array}{cc|cc}
                1 & 0 & -2       & 1         \\
                0 & 1 & \tfrac32 & -\tfrac12
            \end{array}
            \right].
    \end{aligned}
\]

\subsection*{Fitting a line with least squares.}
{\(
    \hat{\alpha}
    =
    \arg\min_{\alpha \in \mathbb{R}^2}
    \left\|
    A\alpha - b
    \right\|^2
    =
    (A^T A)^{-1}A^T b,
    \quad
    A =
    \begin{pmatrix}
        1      & t_1    \\
        \vdots & \vdots \\
        1      & t_m
    \end{pmatrix}
    \)
    \(
    \begin{aligned}
        A                                                                    =\begin{pmatrix}1&0\\1&1\\1&2\end{pmatrix},
        b=\begin{pmatrix}1\\2\\2\end{pmatrix}, \hat{\alpha} =(A^TA)^{-1}A^Tb
        =\begin{pmatrix}\tfrac{7}{6}\\ \tfrac{1}{2}\end{pmatrix}, \hat{b}(t) =\tfrac{7}{6}+\tfrac{1}{2}t.
    \end{aligned}
    \)
}

\subsection*{Forming orthonormal basis via Gram-Schmidt.}
Gram-Schmidt used to construct orthonormal bases. \\ We have linearly indepenedent vectors \(a_1, ..., a_n\) that span a subspace \(S\), then we can construct their orthonormal basis \(q_1,...,q_n\) by:
\\
$\bullet \ q_1 = \frac{a_1}{\|a_1\|}.$
\\
$\bullet$ For $k = 2$, \ldots, $n$ do $q_k' = a_k - \sum_{i=1}^{k-1} (a_k^\top q_i)\, q_i$, \\ $\bullet$ normalise $q_k = \frac{q_k'}{\|q_k'\|}.$

\subsection*{Solving Linear Recurrences via Matrix Diagonalization}
We are given the recurrence relation
\[
    a_n = 5a_{n-1} - 6a_{n-2}, \quad \text{for } n \ge 2.
\]
Using the given formula, we can derive a matrix $M$ such that
\[
    \begin{pmatrix}
        a_{n+1} \\
        a_{n}
    \end{pmatrix}
    =
    \begin{pmatrix}
        5 & -6 \\
        1 & 0
    \end{pmatrix}
    \begin{pmatrix}
        a_{n-1} \\
        a_{n-2}
    \end{pmatrix},
    \quad
    M =
    \begin{pmatrix}
        5 & -6 \\
        1 & 0
    \end{pmatrix},
    \
    \mathbf{g}_n = \begin{pmatrix}
        a_{n+1} \\
        a_n
    \end{pmatrix},
\]
With initial vector
\(
\mathbf{g}_0 =
\begin{pmatrix}
    2 \\
    0
\end{pmatrix},
\
\text{we have }
\mathbf{g}_n = M^n \mathbf{g}_0.
\)
\\
\textbf{Eigenvalues of $M$}
\\
We compute
\(
\det(M - \lambda I)
=
(5 - \lambda)(-\lambda) + 6
=
\lambda^2 - 5\lambda + 6.
\)
\\
Solving,
\(
\lambda^2 - 5\lambda + 6 = 0
\ \Rightarrow \
(\lambda - 3)(\lambda - 2) = 0.
\)
\\
Hence,
\(
\lambda_1 = 3, \quad \lambda_2 = 2.
\)
\\
\textbf{Eigenvectors}
\\
For $\lambda = 3$:
\[
    (M - 3I)
    =
    \begin{pmatrix}
        2 & -6 \\
        1 & -3
    \end{pmatrix}.
\]
Solving $(M - 3I)\mathbf{v} = 0$ gives
\(\
2x - 6y = 0 \ \Rightarrow \ x = 3y,
\)
so EV
\(
\mathbf{v}_1 =
\begin{pmatrix}
    3 \\
    1
\end{pmatrix}.
\)
For $\lambda = 2$:
\[
    (M - 2I)
    =
    \begin{pmatrix}
        3 & -6 \\
        1 & -2
    \end{pmatrix}.
\]
Solving $(M - 2I)\mathbf{v} = 0$ gives
\(
x - 2y = 0 \ \Rightarrow \ x = 2y,
\)
so EV
\(
\mathbf{v}_2 =
\begin{pmatrix}
    2 \\
    1
\end{pmatrix}.
\)
\\
\textbf{Closed Form}
\\
Since $\mathbf{v}_1$ and $\mathbf{v}_2$ are linearly independent, we can write
\[
    \mathbf{g}_0 = \alpha_1 \mathbf{v}_1 + \alpha_2 \mathbf{v}_2.
\]
That is,
\[
    \begin{pmatrix}
        2 \\
        0
    \end{pmatrix}
    =
    \alpha_1
    \begin{pmatrix}
        3 \\
        1
    \end{pmatrix}
    +
    \alpha_2
    \begin{pmatrix}
        2 \\
        1
    \end{pmatrix}.
\]
Solving,
\[
    \alpha_1 = 2, \quad \alpha_2 = -2.
\]
Therefore,
\(
\mathbf{g}_n
=
\alpha_1 3^n \mathbf{v}_1
+
\alpha_2 2^n \mathbf{v}_2
=
2 \cdot
3^n
\begin{pmatrix}
    3 \\
    1
\end{pmatrix}
-
2^{n+1}
\begin{pmatrix}
    2 \\
    1
\end{pmatrix}.
\)
\\
Thus,
\(
a_n = 2\cdot 3^n - 2^{n+1}.
\)

% \[
%     \begin{tikzpicture}[baseline=(M.center)]
%         \matrix (M) [matrix of math nodes,
%             left delimiter=(,
%             right delimiter=),
%             row sep=6mm,
%             column sep=6mm] {
%             1 & 2 & 0 & 1 & 2 \\
%             0 & 0 & 0 & 0 & 3 \\
%             5 & 0 & 0 & 3 & 0 \\
%         };

%         % green arrows
%         \draw[green,->] (M-3-1) -- (M-1-3);
%         \draw[green,->] (M-3-3) -- (M-1-5);

%         % red arrows
%         \draw[red,->] (M-1-2) -- (M-3-4);
%         \draw[red,->] (M-1-4) -- (M-3-2);
%     \end{tikzpicture}
% \]



% \subsection*{Gaussian Elimination (to Row Echelon Form)}
% \textbf{Goal:} Use elementary row operations to transform the augmented matrix $[A|b]$
% into an upper-triangular (row echelon) form, then back-substitute.

% \subsection*{Example (Solve a linear system)}
% Solve:
% \(
% \begin{cases}
%     x + 2y + z = 4 \\
%     2x + y - z = 1 \\
%     3x + 3y = 6
% \end{cases}
% \)
% Augmented matrix:
% \(
% \left[\begin{array}{ccc|c}
%         1 & 2 & 1  & 4 \\
%         2 & 1 & -1 & 1 \\
%         3 & 3 & 0  & 6
%     \end{array}\right]
% \)
% \textbf{Step 1: Eliminate below the first pivot (in column 1).}
% \(
% R_2 \leftarrow R_2 - 2R_1:
% \quad (2,1,-1|1) - 2(1,2,1|4) = (0,-3,-3|-7)
% \)
% \(
% R_3 \leftarrow R_3 - 3R_1:
% \quad (3,3,0|6) - 3(1,2,1|4) = (0,-3,-3|-6)
% \)
% So:
% \(
% \left[\begin{array}{ccc|c}
%         1 & 2  & 1  & 4  \\
%         0 & -3 & -3 & -7 \\
%         0 & -3 & -3 & -6
%     \end{array}\right]
% \)
% \textbf{Step 2: Eliminate below the second pivot (in column 2).}
% \(
% R_3 \leftarrow R_3 - R_2:
% \quad (0,-3,-3|-6) - (0,-3,-3|-7) = (0,0,0|1)
% \)
% So:
% \(
% \left[\begin{array}{ccc|c}
%         1 & 2  & 1  & 4  \\
%         0 & -3 & -3 & -7 \\
%         0 & 0  & 0  & 1
%     \end{array}\right]
% \)
% \textbf{Interpretation:} last row means $0=1$, so the system is \textbf{inconsistent}
% $\Rightarrow$ \textbf{no solution}.

% ---------------------------------------------------------

% \section*{Gauss–Jordan Elimination (to Reduced Row Echelon Form, RREF)}
% \textbf{Goal:} Continue row operations until each pivot column has:
% pivot = 1 and zeros everywhere else in that column. Then read off the solution directly.
% \subsection*{Example (Unique solution)}
% Solve:
% \(
% \begin{cases}
%     x + y + z = 6  \\
%     2x - y + z = 3 \\
%     x + 2y - z = 3
% \end{cases}
% \)
% Augmented matrix:
% \(
% \left[\begin{array}{ccc|c}
%         1 & 1  & 1  & 6 \\
%         2 & -1 & 1  & 3 \\
%         1 & 2  & -1 & 3
%     \end{array}\right]
% \)
% \textbf{Step 1: Eliminate below pivot in column 1.}
% \(
% R_2 \leftarrow R_2 - 2R_1:
% \ (2,-1,1|3) - 2(1,1,1|6) = (0,-3,-1|-9)
% \)
% \(
% R_3 \leftarrow R_3 - R_1:
% \ (1,2,-1|3) - (1,1,1|6) = (0,1,-2|-3)
% \)
% \(
% \left[\begin{array}{ccc|c}
%         1 & 1  & 1  & 6  \\
%         0 & -3 & -1 & -9 \\
%         0 & 1  & -2 & -3
%     \end{array}\right]
% \)
% \textbf{Step 2: Make pivot in row 2 nicer by swapping rows 2 and 3.}
% \(
% R_2 \leftrightarrow R_3
% \Rightarrow
% \left[\begin{array}{ccc|c}
%         1 & 1  & 1  & 6  \\
%         0 & 1  & -2 & -3 \\
%         0 & -3 & -1 & -9
%     \end{array}\right]
% \)
% \textbf{Step 3: Eliminate below pivot in column 2.}
% \(
% R_3 \leftarrow R_3 + 3R_2:
% \ (0,-3,-1|-9) + 3(0,1,-2|-3) = (0,0,-7|-18)
% \)
% \(
% \left[\begin{array}{ccc|c}
%         1 & 1 & 1  & 6   \\
%         0 & 1 & -2 & -3  \\
%         0 & 0 & -7 & -18
%     \end{array}\right]
% \)
% \textbf{Step 4: Make pivot in row 3 equal to 1.}
% \(
% R_3 \leftarrow -\frac{1}{7}R_3:
% \ (0,0,-7|-18)\cdot\left(-\frac{1}{7}\right) = (0,0,1|\tfrac{18}{7})
% \)
% \(
% \left[\begin{array}{ccc|c}
%         1 & 1 & 1  & 6             \\
%         0 & 1 & -2 & -3            \\
%         0 & 0 & 1  & \tfrac{18}{7}
%     \end{array}\right]
% \)
% \textbf{Step 5: Eliminate above pivot in column 3.}
% \(
% R_2 \leftarrow R_2 + 2R_3:
% \ (0,1,-2|-3) + 2(0,0,1|\tfrac{18}{7})
% = (0,1,0|\tfrac{15}{7})
% \)
% \(
% R_1 \leftarrow R_1 - R_3:
% \ (1,1,1|6) - (0,0,1|\tfrac{18}{7})
% = (1,1,0|\tfrac{24}{7})
% \)
% \(
% \left[\begin{array}{ccc|c}
%         1 & 1 & 0 & \tfrac{24}{7} \\
%         0 & 1 & 0 & \tfrac{15}{7} \\
%         0 & 0 & 1 & \tfrac{18}{7}
%     \end{array}\right]
% \)
% \textbf{Step 6: Eliminate above pivot in column 2.}
% \(
% R_1 \leftarrow R_1 - R_2:
% \ (1,1,0|\tfrac{24}{7}) - (0,1,0|\tfrac{15}{7})
% = (1,0,0|\tfrac{9}{7})
% \)
% \(
% \left[\begin{array}{ccc|c}
%         1 & 0 & 0 & \tfrac{9}{7}  \\
%         0 & 1 & 0 & \tfrac{15}{7} \\
%         0 & 0 & 1 & \tfrac{18}{7}
%     \end{array}\right]
% \)
% \textbf{Solution:}
% \(
% x=\frac{9}{7},\quad y=\frac{15}{7},\quad z=\frac{18}{7}.
% \)

% ---------------------------------------------------------

% \section*{Finding the Inverse of a Matrix (Gauss–Jordan)}
% \textbf{Goal:} Start with $[A|I]$ and row-reduce until $[I|A^{-1}]$.
% If you cannot reach $I$ on the left, then $A$ is not invertible.

% \subsection*{Example (Compute $A^{-1}$)}
% Let
% \(
% A=\begin{pmatrix}
%     1 & 2 \\
%     3 & 4
% \end{pmatrix}
% \)
% Form the augmented matrix $[A|I]$:
% \(
% \left[\begin{array}{cc|cc}
%         1 & 2 & 1 & 0 \\
%         3 & 4 & 0 & 1
%     \end{array}\right]
% \)
% \\
% \textbf{Step 1: Eliminate below pivot in column 1.}
% \(
% R_2 \leftarrow R_2 - 3R_1:
% \ (3,4|0,1) - 3(1,2|1,0) = (0,-2|-3,1)
% \)
% \(
% \left[\begin{array}{cc|cc}
%         1 & 2  & 1  & 0 \\
%         0 & -2 & -3 & 1
%     \end{array}\right]
% \)
% \\
% \textbf{Step 2: Make pivot in row 2 equal to 1.}
% \(
% R_2 \leftarrow -\frac{1}{2}R_2:
% \ (0,-2|-3,1)\cdot\left(-\frac{1}{2}\right)=(0,1|\tfrac{3}{2},-\tfrac{1}{2})
% \)
% \(
% \left[\begin{array}{cc|cc}
%         1 & 2 & 1            & 0             \\
%         0 & 1 & \tfrac{3}{2} & -\tfrac{1}{2}
%     \end{array}\right]
% \)
% \\
% \textbf{Step 3: Eliminate above pivot in column 2.}
% \(
% R_1 \leftarrow R_1 - 2R_2:
% \ (1,2|1,0) - 2(0,1|\tfrac{3}{2},-\tfrac{1}{2})
% = (1,0|-2,1)
% \)
% \(
% \left[\begin{array}{cc|cc}
%         1 & 0 & -2           & 1             \\
%         0 & 1 & \tfrac{3}{2} & -\tfrac{1}{2}
%     \end{array}\right]
% \)
% \\
% \textbf{Result:}
% \(
% A^{-1}=
% \begin{pmatrix}
%     -2          & 1            \\
%     \frac{3}{2} & -\frac{1}{2}
% \end{pmatrix}
% \)
% \\
% \textbf{Quick check (optional):}
% \(
% \begin{pmatrix}
%     1 & 2 \\3&4
% \end{pmatrix}
% \begin{pmatrix}
%     -2 & 1 \\ \frac{3}{2}&-\frac{1}{2}
% \end{pmatrix}
% =
% \begin{pmatrix}
%     1 & 0 \\0&1
% \end{pmatrix}
% \\
% \)
